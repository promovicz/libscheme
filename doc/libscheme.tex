\documentstyle{article}

% for final draft
\pagestyle{empty}
\oddsidemargin 0.375in
\evensidemargin 0.375in
\textwidth 5.875in
\topmargin 0in
\textheight 9in
\headheight 0in
\footheight 0in
\headsep 0in
\footskip 0in
\flushbottom

\title{\bf libscheme: Scheme as a C Library}
\author{Brent W. Benson Jr.\\
{\em Harris Computer Systems\/}\\
\verb+Brent.Benson@mail.csd.harris.com+}
\date{}

\begin{document}

\vspace*{0.4in}

\begin{center}
{\Large\bf libscheme: Scheme as a C Library}\\
\vspace{0.2in}
{\large Brent W. Benson Jr.}\\
{\large\em Harris Computer Systems}\\
{\large\tt Brent.Benson@mail.csd.harris.com}
\end{center}

\vspace{0.05in}

\begin{abstract}
  Because of its small size and simplicity, Scheme is often seen as an
  ideal extension or scripting language.  While there are many Scheme
  implementations available, their interfaces are often complex and
  can get in the way of using the implementation as part of a larger
  software product.  The \verb+libscheme+ library makes the Scheme
  language available as a C library.  Its interface is through a
  single C header file and it is easily extended with new primitive
  procedures, new primitive types, and new syntax.  It is portable to
  any system that has an ANSI C compiler and to which Hans Boehm's
  popular conservative garbage collector~\cite{bib:boehm} has been
  ported.  It has been used to build a variety of special purpose data
  manipulation tools, and as an extension language for an ethernet
  monitor.
\end{abstract}

\section{Introduction}

% Give an example of why you want a scripting/extension language

There is a long tradition of scripting languages in the Unix
community, the canonical example being
\verb+/bin/sh+~\cite{bib:bourne}.  Scripting languages allow the
programmer to express programming ideas at a high level, and can also
be designed in such a way that the language interpreter can be
included as an extension language inside of other programs.  When a
program provides a powerful extension language to end users, it often
increases the utility of the program by orders of magnitude (consider
GNU Emacs and GNU Emacs Lisp as an example).  While in recent years
there has been an explosion of general purpose extension and scripting
languages (e.g., Python~\cite{bib:python} and Elk~\cite{bib:elk}), one
language has had a dramatic increase in popularity and seems to have
become the de facto extension language.  That language is
Tcl~\cite{bib:tcl}.

It is the author's opinion that the popularity of Tcl is primarily due
to the ease with which it can be embedded into C applications (its
interface is through a single C header file and a standard C library
archive file) and the ease with which it can be extended with new
primitive commands.  The \verb+libscheme+ library attempts to learn
from Tcl's success by making Scheme~\cite{bib:scheme} available as a C
library and by providing simple ways to extend the language with new
procedures and syntax.  While Scheme is not as convenient as Tcl in
the role of an interactive shell program, it has several advantages
over Tcl with respect to writing scripts:

\begin{enumerate}
\item Lexical Scope
\item Nested procedures
\item A richer set of data types
\item Extensible syntax
\end{enumerate}

In addition, \verb+libscheme+ allows the application writer to extend
the interpreter with new data types that have the same standing as
built in types.  It also provides a conservative garbage collector
that can be used by application and extension writers.

\section{Scheme}

Scheme is a small, lexically scoped dialect of Lisp that is based on
the principle that a programming language should not include
everything but the kitchen sink, but rather it should provide a
framework in which it is easy to build the kitchen sink.

\begin{quotation}
  Programming languages should be designed not by piling feature on
  top of feature, but by removing the weaknesses and restrictions that
  make additional features appear necessary.  Scheme demonstrates that
  a very small number of rules for forming expressions, with no
  restrictions on how they are composed, suffice to form a practical
  and efficient programming language that is flexible enough to
  support most of the major programming paradigms in use
  today.~\cite{bib:scheme} 
\end{quotation}

These properties make Scheme a good general purpose programming
language and also an ideal extension language.  Its conceptual
simplicity allows it to be implemented with a relatively small number
of lines of code, while providing powerful high level programming
constructs such as procedures that can be nested and used as first
class values, and high-level data structures like lists, vectors and
strings.  It also removes the burden of memory management from the
programmer, usually through some sort of garbage collector.  

Scheme supports all major programming paradigms in use today including
functional, procedural, and object oriented.  It scales well from
small applications to large software systems.

\subsection{An Example Procedure}

Let us examine a small Scheme procedure to get a feel for the
language.  The procedure in figure~\ref{fig:split} splits a string of
characters into a list of strings based on a delimiter character.

\begin{figure}[htbp]
\begin{center}
\begin{verbatim}
(define (split-string string delimiter)
  (let ((len (string-length string)))

    ; Collect characters until we reach a delimiter character, 
    ; at which time we extract a field and start looking for
    ; the next delimiter.
    ;
    (define (collect start end)
      (cond
       ((= end len)
        (list (substring string start end)))
       ((char=? (string-ref string end) delimiter)
        (cons (substring string start end)
              (collect (+ end 1) (+ end 1))))
       (else (collect start (+ end 1)))))

    ; Start at the beginning of the string.
    ;
    (collect 0 0)))

\end{verbatim}
\end{center}  
  \caption{A sample Scheme function}
  \label{fig:split}
\end{figure}

In this example we have a top-level definition of the
\verb+split-string+ function and an {\em internal definition\/} of the
\verb+collect+ function.  An internal, or {\em nested\/} function like
\verb+collect+ has access to all lexical variables in the scope of its
definition.  More specifically, \verb+collect+ has access to the
lexical variables \verb+string+, \verb+delimiter+ and \verb+len+.  The
\verb+let+ special form establishes a series of local variables and
initial bindings.  In \verb+split-string+ the \verb+let+ establishes
only the binding of \verb+len+.  The \verb+cond+ special form
evaluates the test in each of its clauses, performing the clauses
action(s) when it finds the first test to succeed.  The \verb+else+
clause of a \verb+cond+ form is executed if no other clause succeeds.
Scheme also has other standard control constructs like \verb+if+ and
\verb+case+.

\begin{verbatim}
> (split-string "brent:WgG6SfAUnX5lQ:5359:100:Brent Benson" #\:)
("brent" "WgG6SfAUnX5lQ" "5359" "100" "Brent Benson")
> 
\end{verbatim}

\subsection{Closures}

Procedures are first class in Scheme, meaning that procedures can be
assigned to variables, passed as arguments to other procedures, be
returned from procedures, and be applied to arbitrary arguments.  When
they are created, procedures ``capture'' variables that are free in
their bodies.\footnote{A variable is free in a procedure if it is not
  contained in the parameter list.} Because of this ``closing over,''
procedures are also known as {\em closures\/} in Scheme.

Closures can be used for a wide variety of purposes.  They can be used
to represent objects in an object-oriented framework, to model actors,
to represent partially computed solutions, etc.

Unnamed procedures are created with the \verb+lambda+ special form.
The existence of unnamed procedures frees the programmer from making
up names for simple helper functions.  For example, many Scheme
implementations have a \verb+sort+ procedure that accepts a list or
vector of elements, and a comparison procedure that is used to
establish an ordering on the elements.  It is often useful to use an
unnamed procedure as the comparison function:

\begin{verbatim}
> (sort '(1 6 3 4) (lambda (n1 n2) (< n1 n2)))
(1 3 4 6)
> (sort #("jim" "brent" "jason" "todd") 
        (lambda (s1 s2) (string<? s1 s2)))
#("brent" "jason" "jim" "todd")
\end{verbatim}

Note that \verb+(1 2 3 4)+ is a list of numbers and 
\verb+#("jim" ...)+ is a vector of strings.

The next example shows the definition of a procedure
\verb+make-counter+ that returns another procedure created with
\verb+lambda+ that closes over the \verb+count+ variable.  

\begin{verbatim}
> (define (make-counter)
    (let ((count 0))
      (lambda ()
        (set! count (+ count 1))
        count)))
make-counter
> (define c1 (make-counter))
c1
> (c1)
1
> (c1)
2
\end{verbatim}

Since no one else has access to \verb+count+ it becomes private to the
closure and can be modified and used within its context.  In this
particular case, \verb+count+ is incremented and its value returned
when the procedure returned from \verb+make-counter+ is called.

\subsection{Syntax}

As you have probably already noticed, Scheme's syntax is Lisp-like.
All function applications are in fully-parenthesized prefix form.
While some find this sort of syntax unwieldy, it has the advantage
that Scheme forms are actually lists which can be easily manipulated
with standard list primitives.  The \verb+libscheme+ library supports
the \verb+defmacro+ special form that can be used by end users to
create new special forms.  A special form is a form that is evaluated
by special rules.  For example, the \verb+if+ special form only
evaluates its ``then'' condition if its test expression evaluates to
true, otherwise it evalutes its ``else'' expression.

These macros are much more powerful than the simple token-based
substitution macros provided by languages like C.

\section{An Application that Uses {\tt libdwarf}}

The \verb+libscheme+ library makes Scheme available as a C library.
Its interface is through a single C header file.  Scheme expressions
can be read from an arbitrary input stream, evaluated with respect to
a particular environment, and printed to an arbitrary output stream.
The user can extend the interpreter by adding bindings to the global
environment.  Each binding can provide a new primitive written in C, a
new syntax form, a new type, a constant, etc.

\subsection{An Example}

DWARF is a full-featured and complex debugging information
format~\cite{bib:dwarf}.  Our example program, \verb+dwarfscheme+, is
an interface that allows the user to browse DWARF information in an
object file by providing stubs to the
\verb+libdwarf+~\cite{bib:libdwarf} library.
Figure~\ref{fig:dialogue} shows a sample \verb+dwarfscheme+ dialogue.

\begin{figure}[htbp]
\begin{center}
\begin{verbatim}
$ ./dwarfscheme
> (define dbg (dwarf-init "a.out"))
dbg
> (define die1 (dwarf-first-die dbg))
die1
> (define (dwarf-print-die die)
    (display (dwarf-tag-string (dwarf-tag die)))
    (newline)
    (for-each 
      (lambda (attr)
        (display "  ")
        (display (dwarf-attr-name-string (dwarf-attr-name attr)))
        (display " ")
        (write (dwarf-attr-value attr))
        (newline))
      (dwarf-attributes die)))
dwarf-print-die
> (dwarf-print-die die1)
DW_TAG_compile_unit
  DW_AT_comp_dir "CX/UX:/jade2/ccg/brent/misc/package/dwarf"
  DW_AT_identifier_case 0
  DW_AT_low_pc 198304
  DW_AT_high_pc 198344
  DW_AT_language 2
  DW_AT_name "t.c"
  DW_AT_producer "Harris C Compiler - Version build_c_7.1p1_03"
  DW_AT_stmt_list 0
#t
> (exit)
$ 
\end{verbatim}
\end{center}
  \caption{{\tt dwarfscheme} dialogue}
  \label{fig:dialogue}
\end{figure}

In this example the user invokes \verb+dwarfscheme+, opens the file
\verb+"a.out"+ for DWARF reading, defines a function for printing out
debugging information entries (DIEs), and prints out the first DIE.
This example shows how \verb+dwarfscheme+ would be used by an end
user.  Next, we will examine the way that the \verb+dwarfscheme+
executable was created using \verb+libscheme+ and the \verb+libdwarf+
libraries.

The program \verb+dwarfscheme+ is an executable that was produced by
linking \verb+libscheme+ with a set of DWARF manipulating primitives,
a read-eval-print loop that initializes the primitives, and the
\verb+libdwarf+ library that is provided as a system library.  The
main routine for \verb+dwarfscheme+ appears in
figure~\ref{fig:simple}.

\begin{figure}[htbp]
\begin{center}
\begin{verbatim}
#include <stdio.h>
#include "scheme.h"
#include "dwarfscheme.h"

main()
{
  Scheme_Env *env;
  Scheme_Object *obj;

  env = scheme_basic_env ();
  init_dwarf (env);
  scheme_default_handler ();
  do
    {
      printf ("> ");
      obj = scheme_read (stdin);
      if (obj == scheme_eof)
        {
          printf ("\n; done\n");
          exit (0);
        }
      obj = scheme_eval (obj, env);
      scheme_write (stdout, obj);
      printf ("\n");
    }
  while ( 1 );
}
\end{verbatim}
\end{center}  
  \caption{{\tt dwarfscheme} read-eval-print loop}
  \label{fig:simple}
\end{figure}

This main routine is a boiler-plate routine that is used when the
application writer wants to make the application a Scheme
read-eval-print loop.  The thing that differentiates the main routines
in different applications is the initializations that are done on the
environment.  In this case, we create a basic environment containing
the standard \verb+libscheme+ bindings, and then add the DWARF
specific bindings to it by calling \verb+init_dwarf()+.  The rest of
the routine takes care of the business of establishing an error
handler, printing out a prompt, reading an expression, evaluating the
expression, and printing out the result.

The application writer can also embed \verb+libscheme+ in an
application that is not structured as a read-eval-print loop.  For
example, at program startup a windowing application might initialize a
\verb+libscheme+ environment, read and evaluate Scheme expressions
representing configuration information from a user configuration file,
and then enter its event loop.  The user might bring up a dialog box
in which she can evaluate Scheme expressions to further configure and
query the system's state.

The major part of the DWARF initialization routine,
\verb+init_dwarf()+ appears in figure~\ref{fig:init}.  It consists of
calls to \verb+scheme_make_type()+ to establish new data types, and
then several calls to \verb+scheme_add_global()+ to add new global
bindings to the environment provided as an argument.  Each call to
\verb+scheme_add_global()+ provides the Scheme name for the global,
the initial value for the variable (in this case a new primitive that
points to its C implementation), and an environment to which the
global should be added.  All routines and variables that are part of
the \verb+dwarfscheme+ interface begin with a \verb+dw+ prefix, while
routines and variables from the system-supplied \verb+libdwarf+
library begin with a \verb+dwarf+ prefix.

\begin{figure}[htbp]
\begin{center}
\begin{verbatim}
static Scheme_Object *dw_debug_type;
static Scheme_Object *dw_die_type;
static Scheme_Object *dw_first_die (int argc, Scheme_Object *argv[]);
...
void
init_dw (Scheme_Env *env)
{
  dw_debug_type = scheme_make_type ("<debug>");
  dw_die_type = scheme_make_type ("<die>");
  dw_attribute_type = scheme_make_type ("<attribute>");
  scheme_add_global ("dwarf-init", 
                     scheme_make_prim (dw_init), env);
  scheme_add_global ("dwarf-first-die", 
                     scheme_make_prim (dw_first_die), env);
  scheme_add_global ("dwarf-next-die", 
                     scheme_make_prim (dw_next_die), env);
  scheme_add_global ("dwarf-tag", 
                     scheme_make_prim (dw_tag), env);
  ...
}
\end{verbatim}
\end{center}  
  \caption{The DWARF primitive initialization routine}
  \label{fig:init}
\end{figure}

This practice of calling an initialization routine with the
environment for each logical piece of code is only a convention, but
is a helpful way of organizing \verb+libscheme+ code.  The
\verb+libscheme+ library itself is organized this way.  Each file
contains an initialization function that establishes that file's
primitives.

A sample primitive is shown in figure~\ref{fig:prim}.  Each
\verb+libscheme+ primitive accepts an argument count and a vector of
evaluated arguments.  Each primitive procedure is responsible for
checking the number and type of its arguments.  All Scheme objects are
represented by the C type \verb+Scheme_Object+ (see
section~\ref{sec:object}).  The types \verb+Dwarf_Debug+ and
\verb+Dwarf_Die+ are foreign to \verb+libscheme+ and are provided by
the \verb+libdwarf+ library.  

\begin{figure}[htbp]
\begin{center}
\begin{verbatim}
static Scheme_Object *
dw_first_die (int argc, Scheme_Object *argv[])
{
  Dwarf_Debug dbg;
  Dwarf_Die die;

  SCHEME_ASSERT ((argc == 1), 
                 "dwarf-first-die: wrong number of args");
  SCHEME_ASSERT (DWARF_DEBUGP (argv[0]), 
                 "dwarf-first-die: arg must be a debug object");
  dbg = (Dwarf_Debug) SCHEME_PTR_VAL (argv[0]);
  die = dwarf_nextdie (dbg, NULL, NULL);
  if (! die)
    {
      return (scheme_false);
    }
  else
    {
      return (dw_make_die (die));
    }
}
\end{verbatim}
\end{center}  
  \caption{A {\tt dwarfscheme} primitive}
  \label{fig:prim}
\end{figure}

The \verb+SCHEME_ASSERT()+ macro asserts that a particular form
evaluates to true, and signals an error otherwise.  The
\verb+dw_first_die()+ routine first checks for the correct number
of arguments, then it checks that the first argument is an object with
type \verb+dw_debug_type+.  Next, it extracts the pointer value
representing the DWARF information in the file from the first
argument, a \verb+Scheme_Object+.  It then calls a \verb+libdwarf+
function, \verb+dwarf_nextdie()+ and returns an appropriate value---a
new \verb+dw_die_type+ object if there is another DIE, the Scheme
false value otherwise.  The \verb+dw_make_die()+ routine accepts a
\verb+Dwarf_Die+ as an argument and returns a \verb+libscheme+ object
of type \verb+dw_die_type+ that contains a pointer to the
\verb+Dwarf_Die+ structure.

Now that we have a feel for the way that \verb+libscheme+ is extended,
we will take a closer look at the design of \verb+libscheme+ itself.

\section{{\tt libscheme} Architecture}

This section describes some specifics of \verb+libscheme+'s
implementation.  An important feature of its design is that beyond a
small kernel of routines for memory management, error handling, and
evaluation, all of its Scheme primitives are implemented in the same
way as non-\verb+libscheme+ extensions.  This is similar to Tcl's
implementation strategy.

\subsection{Object Representation}
\label{sec:object}

Every object in \verb+libscheme+ is an instance of the C structure
\verb+Scheme_Object+.  Each instance of \verb+Scheme_Object+ contains
a pointer to a type object (that also happens to be a
\verb+Scheme_Object+) and two data words.  If an object requires more
than two words of storage or if the object is some other type of
foreign C structure, it is stored in a separate memory location and
pointed to by the \verb+ptr_val+ field.  The actual definition of
\verb+Scheme_Object+ appears in figure~\ref{fig:schemeobj}.

\begin{figure}[htbp]
\begin{center}
\begin{verbatim}
struct Scheme_Object
{
  struct Scheme_Object *type;
  union
    {
      char char_val;
      int int_val;
      double double_val;
      char *string_val;
      void *ptr_val;
      struct Scheme_Object *(*prim_val)
        (int argc, struct Scheme_Object *argv[]);
      struct Scheme_Object *(*syntax_val)
        (struct Scheme_Object *form, struct Scheme_Env *env);
      struct { struct Scheme_Object *car, *cdr; } pair_val;
      struct { int size; struct Scheme_Object **els; } 
               vector_val;
      struct { struct Scheme_Env *env; 
               struct Scheme_Object *code; } closure_val;
    } u;
};
\end{verbatim}
\end{center}
\caption{The definition of {\tt Scheme\_Object}}
\label{fig:schemeobj}
\end{figure}

While many Scheme implementations choose to represent certain special
objects as immediate values (e.g., small integers, characters, the
empty list, etc.) this approach was not used in \verb+libscheme+
because the ``everything is an object with a tag approach'' is simpler
and easier to debug.  A side effect of this decision is that
\verb+libscheme+ code that does heavy small integer arithmetic will
allocate garbage that must be collected, in contrast to higher
performance Scheme implementations that only dynamically allocate very
large integers.

\subsection{Primitive Functions}

Primitive functions in Scheme are implemented as C functions that take
two arguments, an argument count and a vector of
\verb+Scheme_Object+s.  Each primitive is responsible for checking for
the correct number of arguments---allowing maximum flexibility for
procedures of variable arity---and for checking the types of its
arguments.  All arguments to a primitive function are evaluated before
they are passed to the primitive, following Scheme semantics.  If the
application writer wants to create a primitive that doesn't evaluate
its arguments, she must use a syntax primitive.  C functions are
turned into \verb+libscheme+ primitives with the
\verb+scheme_make_prim()+ function that accepts the C function as an
argument and returns a new Scheme object of type
\verb+scheme_prim_type+. 

\subsection{Primitive Syntax}

The user can add new primitive syntax and special forms to
\verb+libscheme+.  A \verb+libscheme+ syntax form is implemented as a
C function that takes two arguments, an expression and an environment
in which the expression should be evaluated.  The form is passed
directly to the syntax form with no evaluation performed.  This allows
the syntax primitive itself to evaluate parts of the form as needed.
Figure~\ref{fig:if} shows the implementation of the \verb+if+ special
form.  Note that \verb+if+ cannot be implemented as a procedure
because it must not evaluate all of its arguments.  The
\verb+scheme_eval()+ function evaluates a \verb+libscheme+ expression
with respect to a particular environment.

\begin{figure}[htbp]
\begin{center}
\begin{verbatim}
static Scheme_Object *
if_syntax (Scheme_Object *form, Scheme_Env *env)
{
  int len;
  Scheme_Object *test, *thenp, *elsep;

  len = scheme_list_length (form);
  SCHEME_ASSERT (((len == 3) || (len == 4)), 
                 "badly formed if statement");
  test = SCHEME_SECOND (form));
  test = scheme_eval (test, env);
  if (test != scheme_false)
    {
      thenp = SCHEME_THIRD (form);
      return (scheme_eval (thenp, env));
    }
  else
    {
      if (len == 4)
        {
          elsep = SCHEME_FOURTH (form);
          return (scheme_eval (elsep, env));
        }
      else
        {
          return (scheme_false);
        }
    }
}
\end{verbatim}    
\end{center}
  \caption{The {\tt if} special form}
  \label{fig:if}
\end{figure}

C functions that represent syntax forms are turned into Scheme objects
by passing them to the \verb+scheme_make_syntax()+ procedure which
returns a new object of type \verb+scheme_syntax_type+.

\subsection{Type Extensions}

Scheme as defined in its standard has the following data types:
boolean, list, symbol, number, character, string, vector, and
procedure.  While Scheme in its current form does not allow the
creation of user-defined types, the \verb+libscheme+ system allows
users to extend the type system with new types by calling the
\verb+scheme_make_type()+ function with a string representing the name
of the new type.  This function returns a type object that can be used
as a type tag in subsequently created objects.  Normally, types are
created in a file's initialization function and objects of the new
type are created using a user-defined constructor function that
allocates and initializes instances of the type.

In figure~\ref{fig:constr} we see the constructor for the
\verb+dw_debug_type+ type from our \verb+dwarfscheme+ example.  It
accepts an object of type \verb+Dwarf_Debug+, a pointer to a C
structure defined in the \verb+libdwarf+ library, allocates a new
\verb+Scheme_Object+, sets the object type, and stores the pointer to
the foreign structure into the \verb+ptr_val+ slot of the object.

\begin{figure}[htbp]
\begin{center}
\begin{verbatim}
static Scheme_Object *
dw_make_debug (Dwarf_Debug dbg)
{
  Scheme_Object *debug;

  debug = scheme_alloc_object ();
  SCHEME_TYPE (debug) = dw_debug_type;
  SCHEME_PTR_VAL (debug) = dbg;
  return (debug);
}
\end{verbatim}
\end{center}
  \caption{An object constructor}
  \label{fig:constr}
\end{figure}

It is often convenient to define a macro that checks whether a
\verb+libscheme+ object is of a specified type.  The macro defined in
\verb+dwarfscheme+ for the DWARF debug object looks like this:
\begin{verbatim}
#define DW_DEBUGP(obj) (SCHEME_TYPE(obj) == dwarf_debug_type)
\end{verbatim}
The `P' at the end of \verb+DW_DEBUGP+ indicates that the macro is
a predicate that returns a true or false value.  All of the builtin
types have type predicate macros of this form (e.g.,
\verb+SCHEME_PAIRP+, \verb+SCHEME_VECTORP+, etc.).

\subsection{Environment Representation}

The state of the interpreter is contained in an object of type
\verb+Scheme_Env+.  The environment contains both global and local
bindings.  The definition of the \verb+Scheme_Env+ structure is shown
in figure~\ref{fig:env}.  The global variable bindings are held in a
hash table.  The local bindings are represented by a vector of
variables (symbols) and a vector of corresponding values.  An
environment that holds local variables points to the enclosing
environment with its \verb+next+ field.  Therefore, variable value
lookup consists of walking the environment chain, looking for a local
variable of the correct name.  If no local binding is found, the
variable is looked for in the global hash table.

\begin{figure}[htbp]
\begin{center}
\begin{verbatim}
struct Scheme_Env
{
  int num_bindings;
  struct Scheme_Object **symbols;
  struct Scheme_Object **values;
  Scheme_Hash_Table *globals;
  struct Scheme_Env *next;
};
\end{verbatim}
\end{center}
  \caption{The {\tt Scheme\_Env} structure}
  \label{fig:env}
\end{figure}

Table~\ref{tab:env} lists the environment manipulation functions.
Unless the user is adding special forms that create variable bindings,
she usually only needs to worry about the \verb+scheme_basic_env()+
and \verb+scheme_add_global()+ functions.  The 
\verb+scheme_basic_env()+ function is used to create a new environment
with the standard Scheme bindings which can then be extended with new
primitives, types, etc. using \verb+scheme_add_global()+.

\begin{table}[htbp]
\begin{center}
  \begin{tabular}{ll}
    \verb+scheme_basic_env ()+ & Return a new \verb+libscheme+ env \\
    \verb+scheme_add_global (name, val, env)+ & Add a new global binding\\
    \verb+scheme_add_frame (syms, vals, env)+ & Add a frame of local bindings \\
    \verb+scheme_pop_frame (env)+ & Pop a local frame \\
    \verb+scheme_lookup_value (sym, env)+ & Lookup the value of a variable\\
    \verb+scheme_lookup_global (sym, env)+ & Lookup the value of a global\\
    \verb+scheme_set_value (sym, val, env)+ & Set the value of a variable\\
  \end{tabular}
\end{center}
  \caption{Environment manipulation functions}
  \label{tab:env}
\end{table}

\subsection{Interpreter Interface}

The \verb+libscheme+ functions that are used for reading, evaluating
and writing expressions are listed in table~\ref{tab:interp}.

\begin{table}[htbp]
\begin{center}
  \begin{tabular}{ll}
    \verb+scheme_read (fp)+ & Read an expression from stream \\
    \verb+scheme_eval (obj, env)+ & Evaluate an object in environment \\
    \verb+scheme_write (fp, obj)+ & Write object in machine readable form \\
    \verb+scheme_display (fp, obj)+ & Write object in human readable form
  \end{tabular}
\end{center}
  \caption{Interpreter functions}
  \label{tab:interp}
\end{table}

These functions can be used in the context of a read-eval-print loop
or called at arbitrary times during program execution.

\subsection{Error Handling}

The \verb+libscheme+ library provides rudimentary error handling
support.  Errors are signaled using the \verb+scheme_signal_error()+
function, or by failing the assertion in a \verb+SCHEME_ASSERT()+
expression.  If the default error handler is installed by calling
\verb+scheme_default_handler()+, then all uncaught errors will print
the error message and \verb+abort()+.  Errors can be caught by
evaluating an expression within a \verb+SCHEME_CATCH_ERROR()+ form.
This macro evaluates its first argument.  If an error occurs during
the execution, the value second argument is returned, otherwise, the
value of the first expression is returned.

\begin{verbatim}
  obj = scheme_read (stdin);
  result = SCHEME_CATCH_ERROR (scheme_eval (obj, env), 0);
  if (result == 0)
     {
       /* error handling code */
     }
  else
     {
       scheme_write (stdout, result);
     }
\end{verbatim}

\subsection{Memory Allocation/Garbage Collection}

The \verb+libscheme+ library uses Hans Boehm and Alan Demers'
conservative garbage collector~\cite{bib:boehm}.  It provides a
replacement for the C library function \verb+malloc()+ called
\verb+GC_malloc()+.  The storage that is allocated by
\verb+GC_malloc()+ is subject to garbage collection.  The collector
uses a conservative approach, meaning that when it scans for
accessible objects it treats anything that could possibly point into
the garbage collected heap as an accessible pointer.  Therefore, it is
possible that an integer or a floating point number that looks like a
pointer into this area could be treated as a root and the storage that
it points to would not be collected.  In practice, this rarely
happens.

Users of \verb+libscheme+ can use the garbage collector in their own
program and are strongly encouraged to make use of it when extending
\verb+libscheme+.  Normally the user can simply call
\verb+scheme_alloc_object()+ to allocate a new \verb+Scheme_Object+,
but occasionally other types of objects need to be allocated
dynamically.

The Boehm/Demers collector is freely available and can run on most any
32-bit Unix machine.

\section{Pros, Cons and Future Work}

The \verb+libscheme+ library is simple to understand and use.  It
builds on the powerful semantic base of the Scheme programming
language.  The library also provides several powerful extensions to
Scheme including an extensible type system and user defined structure
types.  

The \verb+libscheme+ interpreter is not very fast.  The primary reason
is an inefficient function calling sequence that dynamically allocates
storage, creating unnecessary garbage.  This issue is being addressed
and future versions should be a great deal more efficient.  In any
case, \verb+libscheme+ is intended primarily for interactive and
scripting use for which its performance is already adequate.

When compared to a language like Tcl, Scheme is not as well suited for
interactive command processing.  A possible solution is to add a
syntax veneer on top of the parenthetical Scheme syntax for
interactive use.  On the other hand, Scheme's clean and powerful
semantics provide a sizeable advantage over Tcl for writing large
pieces of software.  It also has the advantage of real data types
rather than Tcl's lowest common denominator ``everything is a string''
approach.

The \verb+libscheme+ library has already been used in many small
projects.  The author plans to make \verb+libscheme+ even more useful
by providing a variety of useful bindings including interfaces to the
POSIX system calls, a socket library, a regular expression package,
etc.

\section{Conclusion}

The \verb+libscheme+ library makes Scheme available as a standard C
library and is easily extended with new primitives, types, and syntax.
It runs on most Unix workstations including Harris Nighthawks, Suns,
Intel 386/486 under Linux and OS/2, HP9000s, DECstations and DEC
Alphas running OSF/1, IBM RS/6000s, NeXT machines and many others.
Its simplicity and extensibility lends itself to use as an extension
or scripting language for large systems.  Currently it is being used
by the DNPAP team at Delf University of Technology in the Netherlands
as part of their ethernet monitor, and is being evaluated for use in a
variety of other projects.  The latest version of \verb+libscheme+ can
be obtained from the Scheme Repository, \verb+ftp.cs.indiana.edu+, in
the directory \verb+/pub/imp/+.

\begin{thebibliography}{99}

\bibitem{bib:boehm} Hans Boehm and M. Weiser.  {\em Garbage Collection
    in an Uncooperative Environment\/}.  Software Practice and
  Experience.  pp. 807-820.  September, 1988.

\bibitem{bib:bourne} Stephen Bourne.  {\em An Introduction to the UNIX
    Shell\/}.  Berkeley 4.3 UNIX User's Supplementary Documents.  USENIX
  Association. 

\bibitem{bib:scheme} William Clinger and Jonathan Rees (Editors).
  {\em Revised\/$^4$ Report on the Algorithmic Language Scheme\/}.
  Available by anonymous ftp from \verb+altdorf.ai.mit.edu+.  1991.

\bibitem{bib:elk}  Oliver Laumann.  {\em Reference Manual for the Elk
    Extension Language Kit\/}.  Available by anonymous ftp from
  \verb+tub.cs.tu-berlin.de+.

\bibitem{bib:tcl} John Ousterhout.  {\em Tcl: an Embeddable Command
    Language\/}.  Proceedings of the Winter 1990 USENIX Conference.
  USENIX Association.  1990.

\bibitem{bib:python} Guido van Rossum.  {\em Python Reference
    Manual\/}.  Release 1.0.2.  Available by anonymous ftp from
  \verb+ftp.cwi.nl+.  1994.

\bibitem{bib:dwarf} {\em DWARF Debugging Information Format\/}.  Unix
  International.  Available by anonymous ftp from
  \verb+dg-rtp.dg.com+.  1994.

\bibitem{bib:libdwarf} {\em DWARF Access Library (libdwarf)\/}.  Unix
  International.  1994.

\end{thebibliography}

\section*{The Author}

Brent Benson received a BA in Mathematics from the University of
Rochester 1990 and completed the work for his MS in Computer Science
at the University of New Hampshire in 1992.  He has been a senior
software engineer in the small but feisty compiler group at Harris
Computer Systems since 1992.

\end{document}
